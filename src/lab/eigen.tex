\chapter{Eigenvalues}

In Chapter~\ref{chap:SingularValueDecomposition} on 
Singular Value Decomposition we studied how
transformations resize vectors.
%  the maximum resizing factors
% are the singular values.
In this chapter we return to plotting the image of
the unit circle, to consider another geometric effect.


\section{Turning}
Recall that a linear map has the same effect on 
all vectors that lie on the same line through the origin.
So we can see its
action by plotting what it does to one point on each such line,
using the routine \inlinecode{plot_circle_action}.
\begin{sagecommandline}
sage: load("plot_action.sage")  
sage: q = plot_circle_action(1,0,0,1) 
q.set_axes_range(-3, 3, -3, 3) 
sage: q.save("graphics/eigen000a.pdf") 
sage: p = plot_circle_action(2,0,2,2) 
q.set_axes_range(-3, 3, -3, 3) 
sage: p.save("graphics/eigen000b.pdf")
\end{sagecommandline}
\begin{sagesilent}
load("plot_action.sage")  
q = plot_circle_action(1,0,0,1) 
q.set_axes_range(-3, 3, -3, 3) 
q.save("graphics/eigen000a.pdf", figsize=2.65)
p = plot_circle_action(2,0,2,2) 
p.set_axes_range(-3, 3, -3, 3) 
p.save("graphics/eigen000b.pdf")
\end{sagesilent}
\begin{equation*} \label{gr:firstgraphic}
  \vcenteredhbox{\includegraphics{graphics/eigen000a.pdf}}
  \qquad\mapsunder{\big (\begin{smallmatrix} 2 &0 \\ 2 &2 \end{smallmatrix}\big )}\qquad
  \vcenteredhbox{\includegraphics{graphics/eigen000b.pdf}}
  \tag{$*$}
\end{equation*}
On the left, proceeding counterclockwise, 
the curve begins with red at $(x,y)=(1,0)$.
On the right plot it begins with red at $(1,2)$.
Here are the before and after vectors for that red point.
\begin{sagecommandline}
sage: load("plot_action.sage")  
sage: p = plot_before_after_action(2,0,2,2, [(1,0)], ['red']) 
sage: p.set_axes_range(-0.5, 2, -0.5, 2) 
sage: p.save("graphics/eigen001.pdf", ticks=([1,2],[1,2]))
\end{sagecommandline}
\begin{sagesilent}
load("plot_action.sage")  
p = plot_before_after_action(2,0,2,2, [(1,0)], ['red']) 
p.set_axes_range(-0.5, 2, -0.5, 2) 
p.save("graphics/eigen001.pdf", ticks=([1,2],[1,2]), figsize=2)
\end{sagesilent}
\begin{equation*}
  \vcenteredhbox{\includegraphics{graphics/eigen001.pdf}}
\end{equation*}
On the red vector, the matrix's action is to both resize and rotate.
\begin{sagecommandline}
sage: v = vector(RR, [1,0])
sage: M = matrix(RR, [[2, 0], [2, 2]])
sage: w = M*v
sage: w.norm(), v.norm() 
sage: angle_in_rads = arccos(w*v/(w.norm()*v.norm())) 
sage: angle_in_rads 
\end{sagecommandline}
But the action of a linear map is not uniform across the upper half circle; on
other vectors this same map will resize and rotate by other
amounts.
Here is the effect of the transformation on the point
one sixth of the way around the half circle.
\begin{sagecommandline}
sage: v = vector(RR, [cos(pi/6),sin(pi/6)])
sage: w = M*v
sage: w.norm(), v.norm() 
sage: angle_in_rads = arccos(w*v/(w.norm()*v.norm())) 
sage: angle_in_rads 
\end{sagecommandline}
\begin{sagesilent}
load("plot_action.sage")  
p = plot_before_after_action(2,0,2,2, [(1,0), (cos(pi/6),sin(pi/6))], rainbow(12)) 
p.set_axes_range(-0.5, 2, -0.5, 2.65) 
p.save("graphics/eigen001a.pdf", ticks=([1,2],[1,2]), figsize=2)
\end{sagesilent}
This shows again the before-and-after effect of the map on the red 
input~$(1,0)$, 
and overlays the effect of the map on the orange 
input~$(\cos(\pi/6),\sin(\pi/6))$. 
\begin{equation*}
  \vcenteredhbox{\includegraphics{graphics/eigen001a.pdf}}
\end{equation*}
The orange output is not as long and is not rotated from the input as much.

\Sage{} will compute for us the rotations of the vectors.
At the end of this chapter is the source for a routine 
\inlinecode{plot_color_angles} that
applies the transformation to vectors in the 
half circle, 
computes the angle by which that
vector is rotated, and draws a graph. 
That graph colors the input and output vectors,
making a visual match.
% was:
% angles = find_angles(2,2,0,2,12)
% angles[0:2]
% ticks = ([0,pi/4,pi/2,3*pi/4,pi], [0,pi/2,pi])
% p = scatter_plot(angles, marker='.', ticks=ticks)
% p.set_axes_range(0, pi, 0, pi) 
% p.save("graphics/eigen002.pdf", figsize=2.5, tick_formatter=pi)
\begin{sagecommandline}
sage: load("plot_action.sage")  
sage: p = plot_color_angles(2,0,2,2)
sage: p.set_axes_range(0,pi,0,pi)
sage: p.save("graphics/eigen003.pdf", figsize=3, tick_formatter=[pi,pi])
\end{sagecommandline}
\begin{sagesilent}
load("plot_action.sage")  
p = plot_color_angles(2,0,2,2)
p.set_axes_range(0,pi,0,pi)
p.save("graphics/eigen003.pdf", figsize=3, tick_formatter=[pi,pi])
\end{sagesilent}
% ? won't show axes labels
\begin{center}
  \vcenteredhbox{\includegraphics{graphics/eigen003.pdf}}
\end{center}
The most interesting point on this graph is where the output angle is $0$,
at the input angle of~$\pi/2$.
This is a input vector that the transformation does not turn at all. 
It is resized but not rotated.
This appears in this chapter's first graphic, the one labeled~($*$)
on page~\pageref{gr:firstgraphic}, 
as the green input vector lying on the $y$-axis
that is associated with the green output that also lies on the $y$-axis.




\subsection{Generic matrix}
We can do the same analysis for our usual generic $\nbyn{2}$~matrix.
Here is its action on the unit circle.
\begin{sagecommandline}
sage: load("plot_action.sage")
sage: q = plot_circle_action(1,0,0,1) 
sage: q.set_axes_range(-3, 3, -5, 5) 
sage: q.save("graphics/eigen100a.pdf")
sage: p = plot_circle_action(1,2,3,4) 
sage: p.set_axes_range(-3, 3, -5, 5) 
sage: p.save("graphics/eigen100b.pdf")
\end{sagecommandline}
\begin{sagesilent}
load("plot_action.sage")
q = plot_circle_action(1,0,0,1) 
q.set_axes_range(-3, 3, -5, 5) 
q.save("graphics/eigen100a.pdf")
p = plot_circle_action(1,2,3,4) 
p.set_axes_range(-3, 3, -5, 5) 
p.save("graphics/eigen100b.pdf")
\end{sagesilent}
\begin{equation*}
  \vcenteredhbox{\includegraphics{graphics/eigen100a.pdf}}
  \quad\mapsunder{\big (\begin{smallmatrix} 1 &2 \\ 3 &4 \end{smallmatrix}\big )}\quad
  \vcenteredhbox{\includegraphics{graphics/eigen100b.pdf}}
\end{equation*}
This graphs the angle between the each input vector and its associated output
vector.
% was:
% angles = find_angles(1,2,3,4,50)
% ticks = ([0,pi/4,pi/2,3*pi/4,pi], [0,pi/2,pi])
% p = scatter_plot(angles, marker='.', ticks=ticks)
% p.save("graphics/eigen101.pdf", figsize=2.5, tick_formatter=pi)
\begin{sagecommandline}
sage: load("plot_action.sage")  
sage: p = plot_color_angles(1,2,3,4)
sage: p.set_axes_range(0,pi,-1*pi,pi)
sage: p.save("graphics/eigen101.pdf", tick_formatter=[pi,pi])
\end{sagecommandline}
\begin{sagesilent}
load("plot_action.sage")  
p = plot_color_angles(1,2,3,4)
p.set_axes_range(0,pi,-1*pi,pi)
p.save("graphics/eigen101.pdf", figsize=3, tick_formatter=[pi,pi])
\end{sagesilent}
\begin{equation*}
  \vcenteredhbox{\includegraphics{graphics/eigen101.pdf}}
  \tag{$**$}
\end{equation*}
This graph has two interesting points, where $y=0$ and where 
$y=\pi$.
In both places the vector is not turned at all, only resized.
In the second case
the input vector is on the
same line through the origin as the output,
but it gets rescaled by a negative number.

A vectors that is not turned by a transformation~$t$ but instead
is purely resized 
is an \textit{eigenvector} for~$t$ and the amount by which it is 
resized is the \textit{eigenvalue} for that vector.

For an eigenvalue~$\lambda$, the set of vectors associated with
it is an \textit{eigenspace}.
\begin{sagecommandline}
sage: M = matrix(RDF, [[1, 2], [3, 4]])
sage: evs = M.eigenvectors_right()
sage: evs
sage: evs[0] 
sage: evs[1]
\end{sagecommandline}
\noindent
A matrix has the same eigenvalues whether we are taking multiplication 
by vectors to come from the 
right~$M\vec{v}$ or from the left, $\vec{v}M$.  
But the eigenvectors will be different in general. 
The \Sage{} operation \inlinecode{eigenvectors_right} covers the 
$M\vec{v}=\lambda\vec{v}$ case, and naturally \inlinecode{eigenvectors_left}
covers the other.

When computing the eigenvectors and eigenvalues, \Sage{} gives a list
with two elements.
The first element, \inlinecode{evs[0]}, 
says that the set of vectors with a basis consisting of the single unit
vector with endpoint approximately $(-0.91, 0.42)$  is an
eigenspace associated with the eigenvalue approximately equal to $-0.37$
(we ignore the trailing~$1$ here).
The second element, \inlinecode{evs[1]}, 
says that \Sage{} has found a second eigenspace
whose a basis consisting of the single vector with endpoint around
$(-0.57, -0.82)$ that is associated with the
eigenvalue that is about~$5.37$. 

\Sage{} will tell us which of those vectors is which on the graph 
labeled~($**$).
% Sage says: TypeError: 2.7125795352772983/pi fails to convert into the map's domain Vector space of dimension 2 over Rational Field, but a `pushforward` method is not properly implemented
% \begin{sagecommandline}
% sage: M = matrix(RDF, [[1, 2], [3, 4]])
% sage: evs = M.eigenvectors_left()
% sage: v = vector(RDF, evs[0][1][0])
% sage: angle_v = atan2(v[1], v[0]) 
% sage: n(angle_v/pi) 
% \end{sagecommandline}
(Remember that the \inlinecode{n()} function gives the numerical value of
the argument.)
So the eigenspace listed first is the one associated with the right-hand
of the two interesting points in ($**$), the vector that is turned by an angle
of $\pi$.
That dovetails with the observation that the eigenvalue is a negative number 
because both say that the transformation's action in passing from the
domain to the codomain is to turn the vector around.

\Sage{} can draw before and after pictures for the two eigenvectors.
\begin{sagecommandline}
sage: load("plot_action.sage")  
sage: M = matrix(RDF, [[1, 2], [3, 4]])
sage: evs = M.eigenvectors_right()  
sage: p1 = plot_before_after_action(1,2,3,4, [evs[0][1][0]], ['red']) 
sage: p2 = plot_before_after_action(1,2,3,4, [evs[1][1][0]], ['blue']) 
sage: p = p1+p2
sage: p.set_axes_range(-3, 1, -5, 1) 
sage: p.save("graphics/eigen102.pdf", figsize=3.25)
\end{sagecommandline}
\begin{sagesilent}
load("plot_action.sage")  
M = matrix(RDF, [[1, 2], [3, 4]])
evs = M.eigenvectors_right()  
p1 = plot_before_after_action(1,2,3,4, [evs[0][1][0]], ['red']) 
p2 = plot_before_after_action(1,2,3,4, [evs[1][1][0]], ['blue']) 
p = p1+p2
p.set_axes_range(-3, 1, -5, 1) 
p.save("graphics/eigen102.pdf", figsize=3.25)
\end{sagesilent}
\noindent
This picture has two pairs of before and after vectors, one in blue
and the other in red
(each before vector is a unit vector).
Again we see that 
the after vector is scaled from the before vector, 
in the blue case by the positive factor $\lambda_2\approx 5.37$ and
in the red case by the negative factor $\lambda_1\approx -0.37$.
\begin{equation*}
  \vcenteredhbox{\includegraphics{graphics/eigen102.pdf}}
\end{equation*}



\subsection{Plane rotation}
We know the $\nbyn{2}$~matrix that rotates all vectors counterclockwise.
\begin{equation*}
  R=\begin{mat}
    \cos\theta  &-\sin\theta  \\
    \sin\theta  &\cos\theta
  \end{mat}
\end{equation*}
Here is its action on the upper half circle with $\theta=\pi/3$.
\begin{sagecommandline}
sage: load("plot_action.sage")
sage: q = plot_circle_action(1,0,0,1) 
sage: q.set_axes_range(-2, 2, -2, 2) 
sage: q.save("graphics/eigen200a.pdf")
sage: p = plot_circle_action(cos(pi/3),-sin(pi/3),sin(pi/3),cos(pi/3)) 
sage: p.set_axes_range(-2, 2, -2, 2) 
sage: p.save("graphics/eigen200b.pdf")
\end{sagecommandline}
\begin{sagesilent}
load("plot_action.sage")
q = plot_circle_action(1,0,0,1) 
q.set_axes_range(-1.25, 1.25, -1.25, 1.25) 
q.save("graphics/eigen200a.pdf", figsize=2.5)
p = plot_circle_action(cos(pi/3),-sin(pi/3),sin(pi/3),cos(pi/3))
p.set_axes_range(-1.25, 1.25, -1.25, 1.25) 
p.save("graphics/eigen200b.pdf", figsize=2.5)
\end{sagesilent}
\begin{equation*}
  \vcenteredhbox{\includegraphics{graphics/eigen200a.pdf}}
  \quad\mapsunder{\big (\begin{smallmatrix} \cos(\pi/3) &-\sin(\pi/3) \\ \sin(\pi/3) &\cos(\pi/3) \end{smallmatrix}\big )}\quad
  \vcenteredhbox{\includegraphics{graphics/eigen200b.pdf}}
\end{equation*}
Obviously, the action of this map is to turn every input vector.
Here is the graph giving the angle between each 
input and output.
% was:
% angles = find_angles(1,2,3,4,50)
% ticks = ([0,pi/4,pi/2,3*pi/4,pi], [0,pi/2,pi])
% p = scatter_plot(angles, marker='.', ticks=ticks)
% p.save("graphics/eigen101.pdf", figsize=2.5, tick_formatter=pi)
\begin{sagecommandline}
sage: load("plot_action.sage")  
sage: p = plot_color_angles(cos(pi/3),-sin(pi/3),sin(pi/3),cos(pi/3))
sage: p.set_axes_range(0,pi,-0.25*pi,pi)
sage: p.save("graphics/eigen201.pdf", tick_formatter=[pi,pi])
\end{sagecommandline}
\begin{sagesilent}
load("plot_action.sage")  
p = plot_color_angles(cos(pi/3),-sin(pi/3),sin(pi/3),cos(pi/3))
p.set_axes_range(0,pi,-0.15*pi,pi)
p.save("graphics/eigen201.pdf", figsize=2, tick_formatter=[pi,pi])
\end{sagesilent}
\begin{equation*}
  \vcenteredhbox{\includegraphics{graphics/eigen201.pdf}}
\end{equation*}
This transformation has no real number eigenvalues.


% \subsection{Points}
% Another natural picture to draw is the action of linear maps on 
% some points.

% We start with the action of the linear transformation $\map{t}{\Re^2}{\Re^2}$
% represented with respect to the standard bases by our generic matrix with
% entries $1$, $2$, $3$, and $4$.
% \begin{sagecommandline}
% sage: load("plot_action.sage")
% sage: plot.options['figsize']=4.5
% sage: g = point_grid(3, 3)
% sage: p = plot_point_action(1, 2, 3, 4, g)
% sage: p.save("sagecommandline/plot_action110.pdf")  
% \end{sagecommandline}
% \begin{center}
%   \includegraphics{"sagecommandline/plot_action110.pdf"}
% \end{center}
% This picture shows that the eigenspace associated with the eigenvector
% having the largest absolute value is attractive.
% This is the \textit{power method}:

% \begin{sagecommandline}
% sage: load("plot_action.sage")
% sage: plot.options['figsize']=4.5
% sage: g = point_grid(5, 5)
% sage: p = plot_point_action(2, 2, 0, 2, g)
% sage: p.save("sagecommandline/plot_action111.pdf")  
% \end{sagecommandline}
% \begin{center}
%   \includegraphics{"sagecommandline/plot_action111.pdf"}
% \end{center}




\section{Matrix polynomials}

\Sage{} will find characteristic and minimal polynomials of a matrix.
\begin{sagecommandline}
sage: M =  matrix(RDF, [[1, 2], [3, 4]]) 
sage: poly = M.charpoly()
sage: poly.factor()
sage: poly.roots()
\end{sagecommandline}

The characteristic polynomial and minimal polynomial differ only when 
the characteristic polynomial has repeated roots.
\begin{sagecommandline}
sage: M =  matrix(RDF, [[2, 2, 3], [0, 4, -1], [0, 0, 2]]) 
sage: M.charpoly()
sage: M.minpoly()
sage: M.charpoly().factor()
sage: M.minpoly().factor()
\end{sagecommandline}
Note that \Sage{} has trouble there telling whether $2$ is a repeated 
root.
Do ingan exact calculation using 
rational numbers gets the right answer.  
\begin{sagecommandline}
sage: M =  matrix(QQ, [[2, 2, 3], [0, 4, -1], [0, 0, 2]]) 
sage: M.charpoly()
sage: M.minpoly()
sage: M.charpoly().factor()
sage: M.minpoly().factor()
\end{sagecommandline}

The rotation matrix from above that has no real number eigenvalues does have 
complex eigenvalues.
\begin{sagecommandline}
sage: R = matrix(RDF, [[cos(pi/3), -sin(pi/3)], [sin(pi/3), cos(pi/3)]])
sage: R.minpoly()
sage: R.eigenvalues()
sage: R.eigenvectors_right()
\end{sagecommandline}




\section{Diagonalization and Jordan form}

\Sage{} will tell us if two matrices are similar.
\begin{sagecommandline}
sage: S =  matrix(QQ, [[2, -3], [1, -1]]) 
sage: T =  matrix(QQ, [[0, -1], [1,  1]]) 
sage: S.is_similar(T)
sage: U =  matrix(QQ, [[1, 2], [3,  4]]) 
sage: S.is_similar(U)
\end{sagecommandline}
\noindent
% We can even get a transformation.
% \begin{sagecommandline}
% sage: S =  matrix(QQ, [[2, -3], [1, -1]]) 
% sage: T =  matrix(QQ, [[0, -1], [1,  1]]) 
% sage: S.is_similar(T, transformation=True)
% \end{sagecommandline}

We can determine if a matrix is diagonalizable.
\begin{sagecommandline}
sage: M =  matrix(QQ, [[4, -2], [1, 1]])  
sage: M.is_diagonalizable()
\end{sagecommandline}
\noindent
To diagonalize the matrix put it in Jordan form.
\begin{sagecommandline}
sage: M =  matrix(QQ, [[2, -2, 2], [0, 1, 1], [-4, 8, 3]])  
sage: M.jordan_form()
\end{sagecommandline}
\noindent 
Note the \inlinecode{-+-+-} lines that break the matrix into its component
blocks.

\Sage{} will even give you an appropriate transformation matrix.
\begin{sagecommandline}[d,0,1]
sage: M =  matrix(QQ, [[2, -2, 2], [0, 1, 1], [-4, 8, 3]])  
sage: JF, T = M.jordan_form(transformation=True)
sage: JF
sage: T
sage: T*JF*T^(-1)
\end{sagecommandline}

This is the Jordan form of a non-diagonalizable matrix.
\begin{sagecommandline}
sage: M =  matrix(QQ, [[2, -1], [1, 4]])  
sage: M.jordan_form()
\end{sagecommandline}



\section{Source of plot\_color\_angles}
This routine gathers graphic instances and returns the plot of that list.
\lstinputlisting[firstline=252, lastline=262]{plot_action.sage}

\section{Source of color\_angles\_list}
This finds the angles for the effect of the transformation on the vectors,
and draws a scatter plot of those points.
It draws them in the color of the input vector.
\lstinputlisting[firstline=230, lastline=250]{plot_action.sage}

\section{Source of find\_angles}
This routine uses a formula for the angle between two vectors that 
always gives a positive value, that is, it is the angle without orientiation.
That's fine for purpose here, which is to use the graph to 
roughly locate places where the action of the matrix does not turn the
vector. 
\lstinputlisting[firstline=200, lastline=227]{plot_action.sage}


\section{Source of plot\_before\_after\_action}
The only perhaps unexpected point in this routine and its helper routine
is that if the vector is not mapped very far then the helper
routine does not show an arrow but instead shows a circle.
\lstinputlisting[firstline=155, lastline=195]{plot_action.sage}


\endinput

TODO

1) Does python intro show 
   > x, y = 5, 7
construct?