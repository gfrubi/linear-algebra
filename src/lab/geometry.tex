\chapter{Geometry of Linear Maps}

To illustrate the geometric effect of linear transformations 
we will picture their actions 
of plane transformations on the unit square.
We show the plane 
because it fits on the paper but the principles extend to 
higher-dimensional spaces~$\Re^n$.




%========================================
\section{Lines map to lines}
The prior chapter points out that 
the condition $h(r\cdot \vec{v})=r\cdot h(\vec{v})$ in the definition
of linear map means that
these maps send lines through the origin to lines through the origin.
What about lines not through the origin?
Fix a linear map $\map{h}{\Re^d}{\Re^c}$.
A line in the domain has the form 
$\ell = \set{\vec{m}\cdot s +\vec{b}\suchthat s\in\Re}$
for~$\vec{m}, \vec{b}\in\Re^d$.
Its image
\begin{equation*}
  h(\ell)=\set{h(\vec{m}\cdot s+\vec{b})\suchthat s\in\Re}
  =\set{h(\vec{m})\cdot s+h(\vec{b})\suchthat s\in\Re}
\end{equation*}
is a line in the codomain.
So a linear map sends any line to a line. 

For example, consider the transformation $\map{t}{\Re^2}{\Re^2}$ 
that rotates vectors counterclockwise by 
$\pi/6$~radians.
\begin{equation*}
  \rep{t}{\stdbasis_2,\stdbasis_2}
  =
  \begin{mat}
    \cos(\pi/6)  &-\sin(\pi/6) \\
    \sin(\pi/6)  &\cos(\pi/6)
  \end{mat}
  = 
  \begin{mat}
    \sqrt{3}/2   &-1/2 \\
    1/2          &\sqrt{3}/2
  \end{mat}
\end{equation*}
And consider the line $y=3x+2$.
\begin{equation*}
  \ell=
  \set{\colvec{x \\ y}=\colvec{1 \\ 3}\cdot s+\colvec{0 \\ 2}\suchthat s\in\Re}
\end{equation*}
Under the action of~$t$ 
\begin{equation*}
  \begin{mat}
    \sqrt{3}/2   &-1/2 \\
    1/2          &\sqrt{3}/2
  \end{mat}
  \colvec{1  \\ 3}
  =
  \colvec{(\sqrt{3}-3)/2  \\ (1+3\sqrt{3})/2}
  \qquad
  \begin{mat}
    \sqrt{3}/2    &-1/2 \\
    1/2          &\sqrt{3}/2
  \end{mat}
  \colvec{0  \\ 2}
  =
  \colvec{-1  \\ \sqrt{3}}
\end{equation*}
that line becomes this set.
\begin{equation*}
  t(\ell)
  =
  \set{\colvec{x \\ y}=\colvec{(\sqrt{3}-3)/2 \\ (1+3\sqrt{3})/2}\cdot s
                          +\colvec{-1 \\ \sqrt{3}}\suchthat s\in\Re}
\end{equation*}
\begin{sagecommandline}
sage: s = var('s')
sage: ell = parametric_plot((s, 3*s+2), (s, -10, 10))
sage: d = circle((0,2), 0.1, rgbcolor=(1,0,0))
sage: ell = ell+d
sage: ell.save("graphics/geo000a.pdf")
sage: t_x(s) = ((sqrt(3)-3)/2)*s-1
sage: t_y(s) = ((1+3*sqrt(3))/2)*s+sqrt(3)
sage: t_ell = parametric_plot((t_x(s), t_y(s)), (s, -10, 10))
sage: d = circle((-1,sqrt(3)), 0.1, rgbcolor=(1,0,0))
sage: t_ell = t_ell+d
sage: t_ell.save("graphics/geo000b.pdf")
\end{sagecommandline}
\begin{sagesilent}
s = var('s')
plot.options['figsize'] = 2.5
plot.options['axes_pad'] = 0.05
plot.options['fontsize'] = 7
plot.options['dpi'] = 1200
plot.options['aspect_ratio'] = 1
ell = parametric_plot((s, 3*s+2), (s, -10, 10))
d = circle((0,2), 0.1, rgbcolor=(1,0,0))
ell = ell+d
ell.set_axes_range(-4, 4, -4, 4)
ell.save("graphics/geo000a.pdf", fontsize=7)
t_x(s) = ((sqrt(3)-3)/2)*s-1
t_y(s) = ((1+3*sqrt(3))/2)*s+sqrt(3)
t_ell = parametric_plot((t_x(s), t_y(s)), (s, -10, 10))
d = circle((-1,sqrt(3)), 0.1, rgbcolor=(1,0,0))
t_ell = t_ell+d
t_ell.set_axes_range(-4, 4, -4, 4)
t_ell.save("graphics/geo000b.pdf", fontsize=7)
\end{sagesilent}
Here are the input and output plots.\footnote{%
  As with plots from earlier chapters, we've left out some of the code to 
  set the font size, etc.
  For the full call see this manual's source.
  Also, the limits on the parameter~$s$ of $10$ and~$-10$ are arbitrary, just
  chosen to be large enough that the line segment covers the entire 
  domain and codomain intervals shown, from $-4$ to~$4$.}
\begin{equation*}
  \vcenteredhbox{\includegraphics{graphics/geo000a.pdf}}
  \quad\mapsvia{\makebox[1.5em][c]{\scriptsize $t$}}\quad
  \vcenteredhbox{\includegraphics{graphics/geo000b.pdf}}
\end{equation*}
One example of a rotation is that the vector that ends at $(0,2)$ 
is rotated to the vector that ends at $(-1,\sqrt{3})$.
The endpoints of those vectors are shown as red dots.




%========================================
\section{The unit square}
The prior chapter illustrated
the effect of plane transformations 
$\map{t}{\Re^2}{\Re^2}$ 
by applying them to
the unit circle.
In this chapter we use the unit square.
The observation
that linear maps send lines to lines makes it easy:~we find the
action of a matrix on four corners of the input.
That makes four output corners.
Then we connect those four with line segments.

We start with the unit square.
Note the colors; they will play the same role here as in the prior chapter.
\begin{sagecommandline}
sage: load("plot_action.sage")
sage: p = plot_square_action(1,0,0,1)  # identity matrix
sage: p.save("graphics/geo100.pdf")
\end{sagecommandline}
\begin{sagesilent}
load("plot_action.sage")
p = plot_square_action(1,0,0,1)  # identity matrix
p.set_axes_range(-0.25, 1.5, -0.25, 1.5) 
p.save("graphics/geo100.pdf")
\end{sagesilent}
\begin{center}
  \includegraphics{graphics/geo100.pdf}
\end{center}
The \inlinecode{plot_square_action(a,b,c,d)} routine applies to 
that unit square the 
transformation represented with respect to the standard basis by the 
matrix with entries $a$, $b$, $c$, and~$d$;
the source code listing is at the end of the chapter.
\begin{equation*}
  \begin{mat}
    a &b \\
    c &d
  \end{mat}
  \colvec{x  \\ y}
  =
  \colvec{ax+by \\ cx+dy}
\end{equation*}
This pictures of the effect of the generic matrix.
\begin{sagecommandline}
sage: load("plot_action.sage")
sage: q = plot_square_action(1,0,0,1) 
sage: q.set_axes_range(-1, 4, -1, 7) 
sage: q.save("graphics/geo101a.pdf")
sage: p = plot_square_action(1,2,3,4) 
sage: p.set_axes_range(-1, 4, -1, 7) 
sage: p.save("graphics/geo101b.pdf")
\end{sagecommandline}
% \begin{sagesilent}
% load("plot_action.sage")
% q = plot_square_action(1,0,0,1) 
% q.set_axes_range(-1, 4, -1, 7) 
% q.save("graphics/geo101a.pdf")
% p = plot_square_action(1,2,3,4) 
% p.set_axes_range(-1, 4, -1, 7) 
% p.save("graphics/geo101b.pdf")
% \end{sagesilent}
\begin{equation*}
  \vcenteredhbox{\includegraphics{graphics/geo101a.pdf}}
  \quad\mapsvia{\big (\begin{smallmatrix} 1 &2 \\ 3 &4 \end{smallmatrix}\big )}\quad
  \vcenteredhbox{\includegraphics{graphics/geo101b.pdf}}
\end{equation*}
It transforms the input square into an output parallelogram.

The colors show another effect of the transformations, beyond shape-changing.
Taking the colors in their natural order of red, orange, 
green, and blue,
the domain square has a counterclockwise orientation. 
But the codomain's figure is clockwise. 
So the colors illustrate that this transformation 
reverses orientation.

We can hope to understand complex behavior by 
building up from an understanding of simple behavior.
So we start with the transformation that doubles $x$~components.\footnote{%
  As in other chapters, some of the plots are drawn using 
  some fiddly graphics options that are not shown.
  In this case, the \protect\inlinecode{p.save(...)} has
  the \protect\inlinecode{ticks\_integer=True} option.}
\begin{sagecommandline}
sage: load("plot_action.sage")
sage: q = plot_square_action(1,0,0,1) 
sage: q.set_axes_range(-1, 3, -1, 2) 
sage: q.save("graphics/geo102a.pdf")
sage: p = plot_square_action(2,0,0,1)  
sage: p.set_axes_range(-1, 3, -1, 2) 
sage: p.save("graphics/geo102b.pdf")
\end{sagecommandline}
\begin{sagesilent}
load("plot_action.sage")
q = plot_square_action(1,0,0,1) 
q.set_axes_range(-1, 3, -1, 2) 
q.save("graphics/geo102a.pdf", ticks_integer=True)
p = plot_square_action(2,0,0,1)  
p.set_axes_range(-1, 3, -1, 2) 
p.save("graphics/geo102b.pdf", ticks_integer=True)
\end{sagesilent}
\begin{equation*}
  \vcenteredhbox{\includegraphics{graphics/geo102a.pdf}}
  \quad\mapsvia{\big (\begin{smallmatrix} 2 &0 \\ 0 &1 \end{smallmatrix}\big )}\quad
  \vcenteredhbox{\includegraphics{graphics/geo102b.pdf}}
  \tag{$*$}
\end{equation*}
Linear maps send the zero vector to the zero vector, and the input
square is anchored at the origin, so 
the output shape is also anchored at the origin.
But it has been stretched horizontally\Dash it has the same orientation
as the starting square, but twice the area.

That example illustrates that the behavior associated with diagonal matrices
is simple.
For instance, tripling the $x$~coordinate gives you a similar shape with
three times the area of the starting one.
What if you take $-1$~times the $x$-coordinate?
\begin{sagecommandline}
sage: load("plot_action.sage")
sage: q = plot_square_action(1,0,0,1) 
sage: q.set_axes_range(-2, 2, -1, 2) 
sage: q.save("graphics/geo103a.pdf")
sage: p = plot_square_action(-1,0,0,1) 
sage: p.set_axes_range(-2, 2, -1, 2) 
sage: p.save("graphics/geo103b.pdf")
\end{sagecommandline}
\begin{sagesilent}
load("plot_action.sage")
q = plot_square_action(1,0,0,1) 
q.set_axes_range(-2, 2, -1, 2) 
q.save("graphics/geo103a.pdf", ticks_integer=True)
p = plot_square_action(-1,0,0,1) 
p.set_axes_range(-2, 2, -1, 2) 
p.save("graphics/geo103b.pdf", ticks_integer=True)
\end{sagesilent}
\begin{equation*}
  \vcenteredhbox{\includegraphics{graphics/geo103a.pdf}}
  \quad\mapsvia{\big (\begin{smallmatrix} -1 &0 \\ 0 &1 \end{smallmatrix}\big )}\quad
  \vcenteredhbox{\includegraphics{graphics/geo103b.pdf}}
  \tag{$**$}
\end{equation*}
It changes the orientation.

We say that the shape on the left above has an \textit{oriented area}
of $-1$.
The motivation for taking the area with a sign 
is continuity:~imagine starting with the right-hand figure 
from the example before this one
and slide the orange side in from the right, from $2$ to~$1$, to~$0$ and
then to~$-1$.
The area falls from~$2$ to~$1$, to~$0$, and so we naturally
to assign the figure above an area measure of~$-1$.
The prefix `oriented' is just there to distinguish this idea from the
grade school meaning of
area.
That meaning of area is the absolute value of the oriented area.

The next transformation combines action in two axes, 
tripling the $y$~components and multiplying 
$x$~components by $-1$. 
\begin{sagecommandline}
sage: load("plot_action.sage")
sage: q = plot_square_action(1,0,0,1) 
sage: q.set_axes_range(-2, 2, -1, 4) 
sage: q.save("graphics/geo104a.pdf")
sage: p = plot_square_action(-1,0,0,3) 
sage: p.set_axes_range(-2, 2, -1, 4) 
sage: p.save("graphics/geo104b.pdf")
\end{sagecommandline}
\begin{sagesilent}
load("plot_action.sage")
q = plot_square_action(1,0,0,1) 
q.set_axes_range(-2, 2, -1, 4) 
q.save("graphics/geo104a.pdf", ticks_integer=True)
p = plot_square_action(-1,0,0,3) 
p.set_axes_range(-2, 2, -1, 4) 
p.save("graphics/geo104b.pdf", ticks_integer=True)
\end{sagesilent}
\begin{equation*}
  \vcenteredhbox{\includegraphics{graphics/geo104a.pdf}}
  \quad\mapsvia{\big (\begin{smallmatrix} -1 &0 \\ 0 &3 \end{smallmatrix}\big )}\quad
  \vcenteredhbox{\includegraphics{graphics/geo104b.pdf}}
\end{equation*}
The colors show that this transformation also changes
the orientation, so the new shape has an oriented area of~$-3$.

What if we change the orientation twice?
\begin{sagecommandline}
sage: load("plot_action.sage")
sage: q = plot_square_action(1,0,0,1) 
sage: q.set_axes_range(-3, 2, -4, 2) 
sage: q.save("graphics/geo105a.pdf")
sage: p = plot_square_action(-2,0,0,-3) 
sage: p.set_axes_range(-3, 2, -4, 2) 
sage: p.save("graphics/geo105b.pdf")
\end{sagecommandline}
\begin{sagesilent}
load("plot_action.sage")
q = plot_square_action(1,0,0,1) 
q.set_axes_range(-3, 2, -4, 2) 
q.save("graphics/geo105a.pdf", ticks_integer=True)
p = plot_square_action(-2,0,0,-3) 
p.set_axes_range(-3, 2, -4, 2) 
p.save("graphics/geo105b.pdf", ticks_integer=True)
\end{sagesilent}
The before and after plots
\begin{equation*}
  \vcenteredhbox{\includegraphics{graphics/geo105a.pdf}}
  \quad\mapsvia{\big (\begin{smallmatrix} -2 &0 \\ 0 &-3 \end{smallmatrix}\big )}\quad
  \vcenteredhbox{\includegraphics{graphics/geo105b.pdf}}
\end{equation*}
show that on the right 
the colors come in the same order as they did in the
original shape's counterclockwise
red, orange, green, and then blue.
That is, reversing orientation twice brings you back to the original 
orientation.
The new shape has an oriented area of~$6$.

Next consider the effect of putting off-diagonal entries in the matrix.
\begin{sagecommandline}
sage: load("plot_action.sage")
sage: q = plot_square_action(1,0,0,1) 
sage: q.set_axes_range(-1, 3, -1, 2) 
sage: q.save("graphics/geo106a.pdf")
sage: p = plot_square_action(1,0,2,1) 
sage: p.set_axes_range(-1, 3, -1, 2) 
sage: p.save("graphics/geo106b.pdf")
\end{sagecommandline}
\begin{sagesilent}
load("plot_action.sage")
q = plot_square_action(1,0,0,1) 
q.set_axes_range(-1, 3, -1, 2) 
q.save("graphics/geo106a.pdf", ticks_integer=True)
p = plot_square_action(1,2,0,1) 
p.set_axes_range(-1, 3, -1, 2) 
p.save("graphics/geo106b.pdf", ticks_integer=True)
\end{sagesilent}
\begin{equation*}
  \vcenteredhbox{\includegraphics{graphics/geo106a.pdf}}
  \quad\mapsvia{\big (\begin{smallmatrix} 1 &2 \\ 0 &1 \end{smallmatrix}\big )}\quad
  \vcenteredhbox{\includegraphics{graphics/geo106b.pdf}}
\end{equation*}
This transformation is a \textit{shear}.
The output sides are not at right angles, although still opposite sides are
parallel.
The action
\begin{equation*}
  \begin{mat}
    1  &2  \\
    0  &1
  \end{mat}
  \colvec{x \\ y}
  = \colvec{x+2y \\ y}
\end{equation*}
means that 
a starting vector with a $y$~component of~$1$ gets shifted right by~$2$ while
a starting vector with a $y$~component of~$2$ is shifted right by~$4$, etc.
That is,
vectors are shifted depending on how far they are above or below the
$x$-axis.
The output shape has a base of~$1$
with a height of~$1$, and the orientation is preserved, 
so its oriented area is~$1$.

Putting a nonzero value in the other off-diagonal entry of the matrix,
the lower left, has the same effect except that it shears parallel
to the $y$-axis.
\begin{sagecommandline}
sage: load("plot_action.sage")
sage: q = plot_square_action(1,0,0,1) 
sage: q.set_axes_range(-1, 3, -3, 2) 
sage: q.save("graphics/geo107a.pdf")
sage: p = plot_square_action(1,0,-2,1) 
sage: p.set_axes_range(-1, 3, -3, 2) 
sage: p.save("graphics/geo107b.pdf")
\end{sagecommandline}
\begin{sagesilent}
load("plot_action.sage")
q = plot_square_action(1,0,0,1) 
q.set_axes_range(-1, 3, -3, 2) 
q.save("graphics/geo107a.pdf", ticks_integer=True)
p = plot_square_action(1,0,-2,1) 
p.set_axes_range(-1, 3, -3, 2) 
p.save("graphics/geo107b.pdf", ticks_integer=True)
\end{sagesilent}
\begin{equation*}
  \vcenteredhbox{\includegraphics{graphics/geo107a.pdf}}
  \quad\mapsvia{\big (\begin{smallmatrix} 1 &0 \\ -2 &1 \end{smallmatrix}\big )}\quad
  \vcenteredhbox{\includegraphics{graphics/geo107b.pdf}}
\end{equation*}
The action above
\begin{equation*}
  \colvec{x \\ y} \mapsto \colvec{x \\ -2x+y}
\end{equation*}
means that vectors are shifted depending on how far they are from the
$x$~axis.
For instance, an input vector with an $x$~component of~$1$ is shifted by~$-2$
while if the $x$~component is~$2$ then it is shifted by~$-4$.
The oriented area of the output shape is~$1$.





\section{Determinants}
The book geometrically 
interprets the conditions in the definition of 
a determinant function.
It shows that, in going from the before picture to the after, 
these transformation matrices 
change the oriented area of the input region by a factor that is the
determinant of the matrix.   
In ($*$) the matrix has determinant~$2$ and it doubles the oriented area.
In ($**$) the matrix multiplies the oriented area by~$-1$.

One advantage of starting these before/after pictures with a unit
square is that then the oriented area of the output shape equals the
determinant of the matrix. 

For instance, the matrix
\begin{equation*}
\begin{mat}
  1 &2 \\
  2 &4
\end{mat}
\end{equation*}
is singular so it has determinant~$0$.
\begin{sagecommandline}
sage: load("plot_action.sage")
sage: q = plot_square_action(1,0,0,1) 
sage: q.set_axes_range(-1, 4, -1, 6) 
sage: q.save("graphics/geo200a.pdf")
sage: p = plot_square_action(1,2,2,4) 
sage: p.set_axes_range(-1, 4, -1, 4) 
sage: p.save("graphics/geo200b.pdf")
\end{sagecommandline}
\begin{sagesilent}
load("plot_action.sage")
q = plot_square_action(1,0,0,1) 
q.set_axes_range(-1, 4, -1, 6) 
q.save("graphics/geo200a.pdf")
p = plot_square_action(1,2,2,4) 
p.set_axes_range(-1, 4, -1, 6) 
p.save("graphics/geo200b.pdf")
\end{sagesilent}
\begin{equation*}
  \vcenteredhbox{\includegraphics{graphics/geo200a.pdf}}
  \quad\mapsvia{\big (\begin{smallmatrix} 1 &2 \\ 2 &4 \end{smallmatrix}\big )}\quad
  \vcenteredhbox{\includegraphics{graphics/geo200b.pdf}}
\end{equation*}
The output shape has an area of zero.






\section{Turing's factorization, PA=LDU}
We will now see how the action of any matrix can be decomposed into 
the actions shown above.
This will give us a complete geometric description of any linear map,
that is, you can understand the effect of any transformation by 
breaking down into a sequence of steps that are simple.

Recall that we can do the row operations of Gauss's Method with
matrix multiplication.
For instance, multiplication from the left by this matrix has the effect of the
row operation $2\rho_1+\rho_2$.
\begin{equation*}
  \begin{mat}
    1 &0 &0 \\
    2 &1 &0 \\
    0 &0 &1
  \end{mat}
  \begin{mat}
    3 &1 &4 \\
   -6 &1 &-8 \\
    0 &-3 &2
  \end{mat}
  =
  \begin{mat}
    3 &1  &4 \\ 
    0 &3 &0 \\
    0 &-3  &2
  \end{mat}
\end{equation*}

As described in the book, the  
elementary reduction matrices
come in the three types.
For a matrix~$H$, do row scaling~\( k\rho_i \) 
with \( M_i(k)\,H \), 
swap rows \( \rho_i\leftrightarrow\rho_j \) with \( P_{i,j}\,H \), 
and add a multiple of one row to another
\( k\rho_i+\rho_j \) with \( C_{i,j}(k)\,H \). 
These three
arise from applying a row operation to an identity matrix.
\begin{center}
$I\grstep{k\rho_i}M_i(k)$ where \( k\neq 0 \)
\qquad
\( I\grstep{\rho_i\leftrightarrow\rho_j}P_{i,j} \) where \( i\neq j \)
\qquad
\( I\grstep{k\rho_i+\rho_j}C_{i,j}(k) \) where \( i\neq j \)
\end{center}
For instance, the prior paragraph used the $\nbyn{3}$
matrix $C_{1,2}(2)$.
(We will focus on transformations so we will take all of these
matrices to be same-sized and square.)

Continuing the Gauss's Method reduction started in the matrix equation above,
use $C_{2,3}(-1)$ to perform $-\rho_2+\rho_3$,
producing echelon form.
\begin{equation*}
  \begin{mat}
    1 &0  &0 \\
    0 &1  &0 \\
    0 &-1 &1
  \end{mat}
  \begin{mat}
    1 &0 &0 \\
    2 &1 &0 \\
    0 &0 &1
  \end{mat}
  \begin{mat}
    3 &1 &4 \\
   -6 &1 &-8 \\
    0 &-3 &2
  \end{mat}
  =
  \begin{mat}
    3 &1  &4 \\ 
    0 &3  &0 \\
    0 &0  &2
  \end{mat}
  % \tag{$*$}
\end{equation*}

Observe that if the starting matrix is such that
you don't need any row swapping then 
you can stick with the operations~\( k\rho_i+\rho_j \) where 
$j>i$. 
The elementary
matrices that perform those operations are called lower triangular
since all of their nonzero entries are in the lower left.
(Matrices with all of their nonzero entries in the upper right are 
called upper triangular.)
% Thus the factorization so far is into a product of lower triangular elementary
% matrices and an echelon form matrix.

We can go further.
Next use a diagonal matrix to make the 
leading entries of the nonzero rows of the echelon form matrix into~$1$'s.
\begin{equation*}
  \begin{mat}
    1/3 &0   &0 \\
    0   &1/3 &0 \\
    0   &0   &1/2  
  \end{mat}
  \begin{mat}
    1 &0  &0 \\
    0 &1  &0 \\
    0 &-1 &1
  \end{mat}
  \begin{mat}
    1 &0 &0 \\
    2 &1 &0 \\
    0 &0 &1
  \end{mat}
  \begin{mat}
    3 &1 &4 \\
   -6 &1 &-8 \\
    0 &-3 &2
  \end{mat}
  =
  \begin{mat}
    1 &1/3  &4/3 \\ 
    0 &1  &0 \\
    0 &0  &1
  \end{mat}
  \tag{$*$}
\end{equation*}
Finally, use column operations
to go all the way to a 
block partial identity
matrix.
Here is right-multiplication on the right-hand side of~($*$) to 
add $-1/3$ times the first column to the second column.
\begin{equation*}
  \begin{mat}
    1 &1/3  &4/3 \\ 
    0 &1  &0 \\
    0 &0  &1
  \end{mat}
  \begin{mat}
    1  &-1/3  &0  \\
    0  &1     &0  \\
    0  &0     &1
  \end{mat}
  =
  \begin{mat}
    1 &0  &4 \\ 
    0 &1  &0 \\
    0 &0  &1
  \end{mat}
\end{equation*}
Then adding $-4/3$~times the first column to the third column
leaves an identity matrix.
\begin{equation*}
  \begin{mat}
    3 &1  &4 \\ 
    0 &3  &0 \\
    0 &0  &2
  \end{mat}
  \begin{mat}
    1  &0     &-4/3  \\
    0  &1     &0  \\
    0  &0     &1
  \end{mat}
  =
  \begin{mat}
    1 &0  &0 \\ 
    0 &1  &0 \\
    0 &0  &1
  \end{mat}
   \tag{$**$}
\end{equation*}

In summary, 
if you start with a matrix $A$ that does not require any row swaps then 
you get this matrix equation.
\begin{equation*}
  D_0\cdot L_s\cdots L_2 L_1\cdot A\cdot U_1U_2\cdots U_r = I
\end{equation*}
Here $I$ is a partial identity matrix, 
$D_0$ is a diagonal matrix,
the $L_i$ are lower-triangular row combination matrices,
and the $U_j$ are upper-triangular column combination matrices. 

All of the row operations can be undone
(for instance, $2\rho_1+\rho_2$
is undone with $-2\rho_1+\rho_2$), 
Thus each of those lower triangular matrices
has an inverse.
Likewise, each upper-triangular matrix has an inverse.
That means we can do the algebra to move everything but~$A$ 
to the left side of the equation.
Therefore, 
if you don't need any swaps in a Gauss-Jordan reduction of a matrix~$A$ 
then you get a factorization of the starting matrix
\begin{equation*} 
  A = L_1^{-1}\cdots L_s^{-1}\cdot D\cdot U_r^{-1}\cdots U_1^{-1}
\end{equation*}
where $D$ is the product of $D_0^{-1}$ and the partial identity~$I$, 
which is clearly a diagonal matrix.

To fix the swap issue you can pre-swap:~before 
factoring the starting matrix, first swap its rows
with a permutation matrix~$P$.
\begin{equation*}
  P\cdot A = L_s^{-1}\cdots L_1^{-1}\cdot D\cdot U_r^{-1}\cdots U_1^{-1}
  \tag{$*{*}*$}
\end{equation*}

Now for the coup de gras.
The inverse of a lower triangular matrix is a lower triangular matrix
and the product of lower triangular matrices is 
lower triangular, so you can combine all the $L_i^{-1}$'s in~($*{*}*$) 
into one lower triangular~$L$.
Likewise, you can combine all the $U_i^{-1}$'s into a single 
upper triangular~$U$.
Thus, any matrix factors as $PA=LDU$.

We illustrate with the generic $\nbyn{2}$ transformation of $\Re^2$ represented 
with respect to the standard basis in this way.
\begin{equation*}
  T=
  \begin{mat}
    1 &2 \\
    3 &4
  \end{mat}
\end{equation*}
Gauss's Method is straightforward.
\begin{equation*}
  \begin{mat}
    1 &2 \\
    3 &4
  \end{mat}
  \grstep{-3\rho_1+\rho_2}  
  \begin{mat}
    1 &2 \\
    0 &-2
  \end{mat}
  \grstep{-(1/2)\rho_2}  
  \begin{mat}
    1 &2 \\
    0 &1
  \end{mat}
  \grstep{-2\chi_1+\chi_2}  
  \begin{mat}
    1 &0 \\
    0 &1
  \end{mat}
\end{equation*}
(We use $\chi_i$ for the columns.)
This is the associated factorization.
\begin{equation*}
  \begin{mat}
    1 &2 \\
    3 &4
  \end{mat}
  =
  \begin{mat}
   1 &0 \\
   3 &1 
  \end{mat}
  \begin{mat}
    1 &0 \\
    0 &-2
  \end{mat}
  \begin{mat}
    1 &2 \\
    0 &1
  \end{mat}
\end{equation*}
The product checks out.
\begin{sagecommandline}
sage: L = matrix(QQ, [[1, 0], [3, 1]])
sage: D = matrix(QQ, [[1, 0], [0, -2]])
sage: U = matrix(QQ, [[1, 2], [0, 1]])
sage: L*D*U  
\end{sagecommandline}

We got into this to understand 
the geometric effect of the generic transformation.
\begin{equation*}
  \begin{mat}
   1 &2 \\
   3 &4 
  \end{mat}
  \colvec{x \\ y}
\end{equation*}
\begin{sagecommandline}
sage: load("plot_action.sage")
sage: q = plot_square_action(1,0,0,1) 
sage: q.set_axes_range(-2, 4, -1, 7) 
sage: q.save("graphics/geo300a.pdf")
sage: p = plot_square_action(1,2,3,4) 
sage: p.set_axes_range(-2, 4, -1, 7) 
sage: p.save("graphics/geo300b.pdf")
\end{sagecommandline}
\begin{sagesilent}
load("plot_action.sage")
q = plot_square_action(1,0,0,1) 
q.set_axes_range(-2, 4, -1, 7) 
q.save("graphics/geo300a.pdf")
p = plot_square_action(1,2,3,4) 
p.set_axes_range(-2, 4, -1, 7) 
p.save("graphics/geo300b.pdf")
\end{sagesilent}
\begin{equation*}
  \vcenteredhbox{\includegraphics{graphics/geo300a.pdf}}
  \quad\mapsvia{\big (\begin{smallmatrix} 1 &2 \\ 3 &4 \end{smallmatrix}\big )}\quad
  \vcenteredhbox{\includegraphics{graphics/geo300b.pdf}}
\end{equation*}
\noindent Expand it using the above $LDU$~factorization. 
\begin{equation*}
  \begin{mat}
    1 &2 \\
    3 &4
  \end{mat}
  =
  \begin{mat}
   1 &0 \\
   3 &1 
  \end{mat}
  \begin{mat}
    1 &0 \\
    0 &-2
  \end{mat}
  \begin{mat}
    1 &2 \\
    0 &1
  \end{mat}  
  \colvec{x \\ y}
\end{equation*}
We will look twice at the effects these three.
First we will see the effect of each of $U$, $D$, and~$L$ separately.
Then we will look at the cumulative effect: that of~$U$, then of~$DU$, 
and finally~$LDU$.

The matrix applied first, the rightmost matrix~$U$, 
is a skew parallel to the $x$-axis.
\begin{sagecommandline}
sage: load("plot_action.sage")
sage: q = plot_square_action(1,0,0,1) 
sage: q.set_axes_range(-2, 4, -2.25, 4.25) 
sage: q.save("graphics/geo303a.pdf")
sage: p = plot_square_action(1,2,0,1) 
sage: p.set_axes_range(-2, 4, -2.25, 4.25) 
sage: p.save("graphics/geo303b.pdf")
\end{sagecommandline}
\begin{sagesilent}
load("plot_action.sage")
q = plot_square_action(1,0,0,1) 
q.set_axes_range(-2, 4, -2.25, 4.25) 
q.save("graphics/geo303a.pdf")
p = plot_square_action(1,2,0,1) 
p.set_axes_range(-2, 4, -2.25, 4.25) 
p.save("graphics/geo303b.pdf")
\end{sagesilent}
\begin{equation*}
  \vcenteredhbox{\includegraphics{graphics/geo303a.pdf}}
  \quad\mapsvia{\big (\begin{smallmatrix} 1 &2 \\ 0 &1 \end{smallmatrix}\big )}\quad
  \vcenteredhbox{\includegraphics{graphics/geo303b.pdf}}
\end{equation*}
The second matrix~$D$ rescales and changes the orientation.
\begin{sagecommandline}
sage: load("plot_action.sage")
sage: q = plot_square_action(1,0,0,1) 
sage: q.set_axes_range(-2, 4, -2.25, 4.25) 
sage: q.save("graphics/geo302a.pdf")
sage: p = plot_square_action(1,0,0,-2) 
sage: p.set_axes_range(-2, 4, -2.25, 4.25) 
sage: p.save("graphics/geo302b.pdf")
\end{sagecommandline}
\begin{sagesilent}
load("plot_action.sage")
q = plot_square_action(1,0,0,1) 
q.set_axes_range(-2, 4, -2.25, 4.25) 
q.save("graphics/geo302a.pdf")
p = plot_square_action(1,0,0,-2) 
p.set_axes_range(-2, 4, -2.25, 4.25) 
p.save("graphics/geo302b.pdf")
\end{sagesilent}
\begin{equation*}
  \vcenteredhbox{\includegraphics{graphics/geo302a.pdf}}
  \quad\mapsvia{\big (\begin{smallmatrix} 1 &0 \\ 0 &-2 \end{smallmatrix}\big )}\quad
  \vcenteredhbox{\includegraphics{graphics/geo302b.pdf}}
\end{equation*}

The leftmost matrix, $L$, is a shear parallel to the $y$-axis.
\begin{sagecommandline}
sage: q = plot_square_action(1,0,0,1) 
sage: q.set_axes_range(-2, 4, -2.25, 4.25) 
sage: q.save("graphics/geo301a.pdf")
sage: p = plot_square_action(1,0,3,1) 
sage: p.set_axes_range(-2, 4, -2.25, 4.25) 
sage: p.save("graphics/geo301b.pdf")
\end{sagecommandline}
\begin{sagesilent}
load("plot_action.sage")
q = plot_square_action(1,0,0,1) 
q.set_axes_range(-2, 4, -2.25, 4.25) 
q.save("graphics/geo301a.pdf")
p = plot_square_action(1,0,3,1) 
p.set_axes_range(-2, 4, -2.25, 4.25) 
p.save("graphics/geo301b.pdf")
\end{sagesilent}
\begin{equation*}
  \vcenteredhbox{\includegraphics{graphics/geo301a.pdf}}
  \quad\mapsvia{\big (\begin{smallmatrix} 1 &0 \\ 3 &1 \end{smallmatrix}\big )}\quad
  \vcenteredhbox{\includegraphics{graphics/geo301b.pdf}}
\end{equation*}
  
The lower-triangular and upper-triangular matrices do not
change orientation.
Any orientation changing in $LDU$ happens via diagonal entries that are 
negative.
(\textit{Comment.}
However, if a matrix requires row swaps, $PA=LDU$, then the situation is 
more subtle.
Each row swap toggles the orientation, from counterclockwise to clockwise or
from clockwise to counterclockwise.
Thus
if the permutation matrix requires an odd number of swaps 
then it changes the orientation, 
but with an even number of swaps it leaves the orientation the same.)

We finish our study of $LDU\,\vec{v}$ by considering the cumulative effect.
Consider in order the actions of $U$, then~$DU$, and then~$LDU$.
For example, here is the cumulative effect of the maps on the 
unit square's upper right corner.
\begin{equation*}
  \colvec{1 \\ 1}
  \mapsunder{U}
  \colvec{3 \\ 1}
  \mapsunder{D}
  \colvec{3 \\ -2}
  \mapsunder{L}
  \colvec{3 \\ 7}
\end{equation*}

\begin{sagecommandline}
sage: L = matrix(QQ, [[1, 0], [3, 1]])
sage: D = matrix(QQ, [[1, 0], [0, -2]])
sage: U = matrix(QQ, [[1, 2], [0, 1]])
sage: DU = D*U
sage: LDU = L*DU  
sage: load("plot_action.sage")
sage: p = plot_square_action(1,0,0,1) 
sage: p.set_axes_range(-4, 4, -2, 7) 
sage: p.save("graphics/geo304a.pdf")
sage: p = plot_square_action(U[0][0], U[0][1], U[1][0], U[1][1]) 
sage: p.set_axes_range(-4, 4, -2, 7) 
sage: p.save("graphics/geo304b.pdf")
sage: p = plot_square_action(DU[0][0], DU[0][1], DU[1][0], DU[1][1]) 
sage: p.set_axes_range(-4, 4, -2, 7) 
sage: p.save("graphics/geo304c.pdf")
sage: p = plot_square_action(LDU[0][0], LDU[0][1], LDU[1][0],LDU[1][1]) 
sage: p.set_axes_range(-4, 4, -2, 7) 
sage: p.save("graphics/geo304d.pdf")
\end{sagecommandline}
\begin{sagesilent}
L = matrix(QQ, [[1, 0], [3, 1]])
D = matrix(QQ, [[1, 0], [0, -2]])
U = matrix(QQ, [[1, 2], [0, 1]])
DU = D*U
LDU = L*DU  
load("plot_action.sage")
p = plot_square_action(1,0,0,1) 
p.set_axes_range(-4, 4, -2, 7) 
p.save("graphics/geo304a.pdf")
p = plot_square_action(U[0][0], U[0][1], U[1][0], U[1][1]) 
p.set_axes_range(-4, 4, -2, 7) 
p.save("graphics/geo304b.pdf")
p = plot_square_action(DU[0][0], DU[0][1], DU[1][0], DU[1][1]) 
p.set_axes_range(-4, 4, -2, 7) 
p.save("graphics/geo304c.pdf")
p = plot_square_action(LDU[0][0], LDU[0][1], LDU[1][0], LDU[1][1]) 
p.set_axes_range(-4, 4, -2, 7) 
p.save("graphics/geo304d.pdf")
\end{sagesilent}
\begin{center}
  \begin{tabular}{rcl}
    \vcenteredhbox{\includegraphics{graphics/geo304a.pdf}}
    &$\mapsvia{\big (\begin{smallmatrix} 1 &2 \\ 0 &1 \end{smallmatrix}\big )}$
    &\vcenteredhbox{\includegraphics{graphics/geo304b.pdf}}  \\
    &$\mapsvia{\big (\begin{smallmatrix} 1 &0 \\ 0 &-2 \end{smallmatrix}\big )}$
    &\vcenteredhbox{\includegraphics{graphics/geo304c.pdf}}  \\
    &$\mapsvia{\big (\begin{smallmatrix} 1 &0 \\ 3 &1 \end{smallmatrix}\big )}$
    &\vcenteredhbox{\includegraphics{graphics/geo304d.pdf}} 
  \end{tabular} 
\end{center}



\section{Source of plot\_action.sage}
The \inlinecode{plot_square_action}
routine takes the four entries of the $\nbyn{2}$
matrix and returns a list of graphics.
Most of the heavy lifting here
is done by the helper \inlinecode{color_square_list}, 
given below.
\lstinputlisting[firstline=92,lastline=103]{plot_action.sage}

There are two technical points about the helper routine that could do with 
explanation. 
First is \inlinecode{ZORDER}, which determines the order in which 
\Sage{} plots things, and here we want the unit square to be plotted
after the axes (so its colors will be visible).
The second is that the way line segments butt against each other is ugly 
so we cover that with a dot.
\lstinputlisting[firstline=53,lastline=90]{plot_action.sage}


\endinput


TODO:
