\chapter{\python{} and \Sage{}}

\Sage{} is based on \python{} so we'll start with that.
\python{} is a popular language\footnote{Written by a fan of
Monty Python.} 
with a simple style and powerful libraries.
It is often used for scripting, as a glue to bring together separate parts.
\Sage{} uses it in this way.

Here you will see enough \python{} to get started; this assumes 
that you already know some programming.
For a more comprehensive introduction see \python's excellent tutorial,
\cite{PythonTeam19b}.



\subsection{Basics}
Start \python, for instance by giving the command
\inlinecode{python3}
at a command line.
You'll get a couple of lines of initial
information followed by a prompt, three greater-than
characters.
\begin{lstlisting}[style=python]
>>>   
\end{lstlisting}
This is the \python{} interpreter.
If you type 
\python{} code and \keyboardkey{Enter} then the system
will read your code, evaluate it, and print the result.
We will see below how to write and run whole programs
but for now we will experiment in the interpreter.
When you are done you can 
leave this prompt with \textit{\keyboardkey{Ctrl}-D}.

Try entering these expressions.
\python{} responds with the results shown
(double star is exponentiation).
\begin{pythonconsole}
2 - (-1)
1 + 2*3
2**3  
\end{pythonconsole}

Part of \python's appeal is that simple things tend to be simple.
Here is how you print something to the screen.\footnote{%
  \protect\python{} comes in two versions.
  Version~3 is what we use.
  Version~2 is quite old but you may still see it, for example if
  you look in a search engine for code.
  One difference between the two is that in version~2 the
  print statement did not have parentheses.}
\begin{pythonconsole}
print(1, "plus", 2, "equals", 3)
\end{pythonconsole}
% Often you can debug just by putting in commands to print things, 
% and having a straightforward print operator helps with that. 

As in other computer languages, 
variables give you a named place to keep values.
The first line below puts $1$ in the place called \inlinecode{i}
and the second line involves fetching that $1$.
\begin{pythonconsole}
i = 1
i + 1
\end{pythonconsole}
In some programming languages you must declare the `type' of a variable
before you use it; for instance above you would have to declare 
that \inlinecode{i} is an integer before you could set \inlinecode{i=1}.
In contrast, \python{} deduces the type 
based on what you do to it\Dash above we assigned $1$ to it
so \python{} figured that \inlinecode{i} must be a place to put an integer.
If we change how we use a variable then \python{} will 
refigure; here we change what is kept in $x$ from an integer to a string.
\begin{pythonconsole}
x = 1
x
x = 'a'
x
\end{pythonconsole}

You can do multiple things on a single line.
\begin{pythonconsole}
first_day, last_day = 0, 365
first_day
last_day   
\end{pythonconsole}
\python{} computes the right side, left to right, and then assigns 
those values to the variables on the left.

If you do something not allowed then 
\python{} complains, signaling its unhappiness by 
raising an exception and in the final line giving an error 
message.
\begin{pythonconsole}
'a' + 1
\end{pythonconsole}
% In an error message, often the bottom line is the most useful.

Besides integers, and also rational numbers, 
\python{} gives you real numbers and complex numbers.\footnote{%
  These operations
  are done with floating points, a system 
  built into your computer's hardware that
  models the real numbers.
  In the example the distinction leaks through
  since the decimal for $2/3$ is not perfectly accurate.
  For that matter, \protect\python's integers aren't the integers  
  from grade school since there is a largest one 
  (try giving \python{} the 
  command \protect\inlinecode{import sys} followed by 
  \protect\inlinecode{sys.maxint}). 
  While all of this is fascinating\protect\Dash
  see \protect\cite{PythonTeam19a} and \protect\cite{Goldberg91}\protect\Dash
  we shall often ignore these issues, just to keep the focus on 
  linear algebra.}
\begin{pythonconsole}
2/3
5.774 * 3
(3+2j) - (1-4j)
\end{pythonconsole}
As engineers do, \python{} uses the letter~$j$ for the square
root of~$-1$, not~$i$ as is traditional in Mathematics.
(Although \protect\Sage{} lets you use~$i$.)

Variables can also represent truth values.
\begin{pythonconsole}
yankees_stink = True
yankees_stink
\end{pythonconsole}
You need the initial capital letter\Dash
\inlinecode{True}
or \inlinecode{False}, not
\inlinecode{true}
or \inlinecode{false}.
 
Above we saw a couple of strings consisting of text between quotes.
(They just had a single character, \inlinecode{'a'},
but \python{} doesn't worry about the difference between a character
and a length one string.)
You can use either single quotes or double quotes, as long as you use
the same at both ends of the string. 
Here \inlinecode{x} and \inlinecode{y}
are double-quoted, which makes sense because each contains
an apostrophe. 
\begin{pythonconsole}
x = "I'm Popeye the sailor man"
y = "I yam what I yam and that's all what I yam"
x + ', ' + y + '.'
\end{pythonconsole}
The last one shows that~\inlinecode{+} 
concatenates strings. 
% and also shows that you can mix single-quoted strings
% with double-quoted ones.

To get a newline you can use use slash-n, \inlinecode{\\n}, 
inside a double-quoted string.
If you need a lot of newlines so the slash-n would be a burden then
you can put explicit line breaks inside a string marked by triple quotes.
\begin{pythonconsole}
a = """THE ROAD TO WISDOM
 
The road to wisdom?
-- Well, it's plain
and simple to express:
Err
and err
and err again
but less
and less
and less. --Piet Hein"""
\end{pythonconsole} 
The three dots that start the lines after the first
is \python's
interpreter prompting you that what you have typed is not complete,
that you've started a string but not yet finished it.

A \python{} \textit{dictionary} is a finite function.
That is, it is a finite
set of ordered pairs $\langle\textit{key},\textit{value}\rangle$ subject 
to the restriction that no key can appear twice.
\python{} programmers often use a dictionary as a simple database.
\begin{pythonconsole}
english_words = {'one': 1, 'two': 2, 'three': 3}
english_words['one']
english_words['four'] = 4  
english_words
english_words['two'] = -1
english_words
\end{pythonconsole} 
Dictionaries are central to \python, in part because looking up values 
in a dictionary is very fast, due to the way that they are 
organized internally.
But because of that internal organization,
if you ask to see a dictionary then \python{} may show its pairs in 
a jumbled-up order. % This changes in 3.6 and 3.7 in Cython, but discussion of insertion order seems out of place

While dictionaries are unordered, a \python{} list is ordered.
\begin{pythonconsole}
a = ['alpha', 'beta', 'gamma']
a
b = ['delta']
c = []
\end{pythonconsole}
Lists can contain anything, including other lists.
\begin{pythonconsole}
x = 4
a = ['alpha', [True, x]]
a
len(a)
\end{pythonconsole}
Get an element from a list by specifying its index, its place in the list,
inside square brackets.
Lists indices are zero-based, that is, the initial element of the
list is numbered~$0$.
Count from the back by using negative indices.
\begin{pythonconsole}
a = ['alpha', 'beta', 'gamma']
a[0]
a[0] = 'Alpha'
a[0]
a[1]
a[-1]
\end{pythonconsole}
Specifying two indices separated by a colon gets a \textit{slice} 
of the list. 
\begin{pythonconsole} 
a = ['alpha', 'beta', 'gamma', 'delta']
a[1:3]
a[1:-1]
\end{pythonconsole}
You can lengthen a list.
\begin{pythonconsole}
c = ['delta']
c.append('<End>')
c
a = ['alpha', 'beta', 'gamma']
a.insert(0,'<Start>')
a+c
\end{pythonconsole}

A \textit{tuple} is like a list in that it is ordered.
\begin{pythonconsole}
a = ('fee', 'fie', 'foe', 'fum')
a[0]
\end{pythonconsole}
However a tuple is unlike a list 
in that it is not mutable, it cannot change.
\begin{pythonconsole}
a = ('fee', 'fie', 'foe', 'fum')
a[0] = 'phooey'
\end{pythonconsole}
One reason that non-mutability is useful is that 
tuples can be keys in dictionaries while lists cannot be keys.
\begin{pythonconsole}
a = ['ke1az', 5418]
b = ('ke1az', 5418)
d = {a: 'active'}
d = {b: 'active'}
d
\end{pythonconsole}

\python{} has a special value, \inlinecode{None}, that
means something has no sensible value.
For instance, if your program keeps track of a person's address and
includes a variable \inlinecode{apartment_num} then   
for a person who does not live in an
apartment you should say \inlinecode{apartment_num = None}.



\subsection{Flow of control}
\python{} supports the traditional ways of changing the order of 
statement execution, with a twist.
The twist is that while many languages use braces or some other syntax to
mark a block of code, \python{} uses indentation.
(Always indent with four spaces, not tabs.)
\begin{pythonconsole}
x = 4
if (x == 0):
    delta = 1
else:
    delta = 0

delta
\end{pythonconsole}
Here, \python{} sets \inlinecode{delta} to  $1$ if $x$
equals $0$, otherwise \python{} sets \inlinecode{delta} to~$0$. 

After \inlinecode{delta = 0} you must enter 
a blank line so that \python{} knows there is nothing more in that block.
Notice also that double equals, \inlinecode{==}, means `is equal to'. 
% (We cannot use just a single equals because, as we have already seen, 
% \inlinecode{x = 0} 
% means ``$x$ is assigned the value~$0$.'') 
For `not equal' use\inlinecode{!=}. 

\python{} has two more variants on the \inlinecode{if} statement.
It could have a single branch
\begin{pythonconsole}
x = 0
if (x == 0):
    print("Division by x not allowed.")

\end{pythonconsole}
or it could have more than two branches.
\begin{pythonconsole}
x = 2
if (x == 0):
    sgn = 0
elif (x > 0):
    sgn = 1
else:
    sgn = -1

sgn
\end{pythonconsole}

Computers excel at iteration, at looping through a sequence of statements.
A \inlinecode{for} loop often involves a \inlinecode{range}.
\begin{pythonconsole}
for i in range(4):
    print(i, "squared is", i**2)

\end{pythonconsole}
By default the lowest number produced by \inlinecode{range} is $0$, which
fits with the fact that list numbering is zero-based.
Change the initial number by calling \inlinecode{range} 
with two arguments.
\begin{pythonconsole}
for i in range(1, 4):
    print(i, "cubed is", i**3)

\end{pythonconsole}
The highest number produced by \inlinecode{range} is one less than the 
argument, so that the highest number
given by \inlinecode{range(4)} or \inlinecode{range(1,4)} is $3$.
This is convenient because it makes the 
combination \inlinecode{range(4)+range(4,10)} give 
the same list as \inlinecode{range(10)}.

You can iterate over a list by working with element indices.
\begin{pythonconsole}
x = [4, 0, 3, 0]
for i in range(len(x)):
    if (x[i] == 0):
        print("item", i, "is zero")
    else:
        print("item", i, "is nonzero")

\end{pythonconsole}
However, an experienced  \python{} person who was not trying to illustrate 
the \inlinecode{range} command
would replace \inlinecode{for i in range(len(x))}
with \inlinecode{for c in x} because
\inlinecode{for} can iterate over any sequence, not just
a list of numbers.
Here is a simple text formatter that fills a line with words until
the line would exceed $40$ characters, and then
prints that line and starts a new one.
\begin{pythonconsole}
quote = """Victorious warriors win first and then go to war, 
while defeated warriors go to war first and then seek to win. 
--Sun Tzu"""
line, line_length = [], 0
for wd in quote.split():
    if line_length+len(wd) > 40:
        print(" ".join(line))
        line, line_length = [], 0
    line_length = line_length + len(wd) + 1
    line.append(wd)

if line:
    print(" ".join(line))

\end{pythonconsole}
Here we see some new string operations.
The \inlinecode{quote.split()} divides the string into separate words while
\inlinecode{" ".join(line)} makes a string of the words in the list 
\inlinecode{line} by putting spaces between them.
The final \inlinecode{if line:} ejects what's left of the line
when the quote is finished.

A \inlinecode{for} loop is designed to execute a certain
number of times.
If you have a loop that will run an uncertain number of times then
you can use \inlinecode{while}.\footnote{The 
\protect\textit{Collatz conjecture} is 
that this loop will 
terminate for any starting~$n$.
No one knows if the conjecture is true.}
\begin{pythonconsole}
n, i = 12, 0
while (n != 1):
    if (n % 2 == 0):  
        n = n // 2
    else:
        n = 3*n + 1
    i = i + 1
    print("i=", i, "n=", n)

\end{pythonconsole}
The \inlinecode{//} operation is division between two integers, which
throws away the remainder.

Exit from a loop immediately with the \inlinecode{break} command.
\begin{pythonconsole}
for i in range(10):
    if (i == 3):
        break
    print("inside the loop: i is", i)

\end{pythonconsole}

A common loop construct is to perform some action
on each list element.
\python{} has a shortcut for this, called list comprehension.
\begin{pythonconsole}
a = [2**i for i in range(4)]
a
[i-1 for i in a]
\end{pythonconsole}



\subsection{Functions}
You can make your own functions.
This implements the quadratic formula.\footnote{%
  This code is naive, for instance in that it does not address the
  potenital floating point issues.}
\begin{pythonconsole}
def quad_formula(a, b, c):
    """Find roots of a quadratic 
      a, b, c  real numbers  Coefficients, as in ax^2 + bx + c
    """
    discriminant = (b**2 - 4*a*c)**(0.5)  # power 0.5 is square root
    r1 = (-1*b-discriminant) / (2.0*a)
    r2 = (-1*b+discriminant) / (2.0*a)
    return (r1, r2)

quad_formula(1, -6, 5)
quad_formula(1, 2, 1)
\end{pythonconsole}
When you write programs in \python{}, most of what you write
is inside of functions. 

Functions always return something; 
if a function never executes a \inlinecode{return} then it will
return the value \inlinecode{None}.
They can also contain multiple \inlinecode{return} statements, for instance 
one for an \inlinecode{if} branch and one for an \inlinecode{else}.

About comments:~the \inlinecode{quad_formula} function has two kinds of 
comments.
Inside the code, inline comments start with a hash mark, \inlinecode{\#}
(programmers often write comments that take up an entire line, by starting 
that line with a hash). 
The second kind is that
after the \inlinecode{def} line is a triple-quoted string that
briefly describes
what the function does and lists its parameters.
% Although \python{} does not require this, in practice every function
% of more than two or three lines should have such a comment.

At the end of the \inlinecode{def} line, in parentheses, are
the function's parameters. 
These take the values 
passed into the function by the caller.
Functions can also have optional parameters that have a default value.
\begin{pythonconsole}
def greet(name="Sir or Madam"):
    """Say hello.
      name  string  Person's name
    """
    print("Hello", name+".")

greet("Fred")
greet()
\end{pythonconsole}
\Sage{}  heavily uses this feature of \python{}.







\subsection{Objects and modules}
In Mathematics, a structure is a set of things that have associated operations.
For example, a vector space is a set with the operations
of addition and scalar multiplication.
\python{} is object-oriented, which means that we can similarly bundle
together data and actions (in this context the actions, the functions, 
are called methods).
\begin{pythonconsole}
class DatabaseRecord(object):
    """Hold information about a person.
    """
    def __init__(self, name, age):
        self.name = name
        self.age = age
    def salutation(self):
        print("Dear", self.name)

a = DatabaseRecord("Jim", 61)
a.name
a.age
a.salutation()
b = DatabaseRecord("Fred", 16)
b.salutation()
\end{pythonconsole}
This creates two instances of 
\inlinecode{DatabaseRecord}, instance \inlinecode{a} and
instance~\inlinecode{b}.
To get the age data for instance~\inlinecode{a}
you write 
\inlinecode{a.age}.
(The \protect\inlinecode{self} variable 
can  be puzzling.
Suppose that at the prompt you type \inlinecode{a.name="James"}.
Then you've used the name~\inlinecode{a} 
for the instance so that the computer knows where to
make the change.
In contrast,
inside the \inlinecode{class} description code there isn't any fixed instance. 
So \inlinecode{self} gives you a way to refer to the current instance. 
For example, if you are working with the instance \inlinecode{a} then 
in the \inlinecode{salutation} the \inlinecode{self.name} refers
to \inlinecode{a.name}.)

You won't be writing your own classes here 
but you will be using ones from
the libraries of code that others have written, including the
code for \Sage, so you must know how to use 
other peoples's classes. 
For instance, for mathematics \python{} has a library, 
sometimes called a module.
\begin{pythonconsole}
import math
math.pi
math.factorial(5)
math.cos(math.pi)
\end{pythonconsole}
The \inlinecode{import} statement makes the module's
contents available.

Another module does random numbers.
\begin{pythonconsole}
import random
print(random.randint(1,10))

\end{pythonconsole}



\subsection{Programs}
The read-eval-print loop is great for small experiments but
for code with more than a few lines, or code that you want to
keep, you 
want to put your work in a separate file and run it as a standalone program.

To write the code, use a text editor.
Don't use a word processor; instead an editor has the right features for 
writing programs and will output your work in a form suitable for
input to \python .
Use one with support for \python{}, including
automatic indentation and  
syntax highlighting, where the editor colors your code to make it easier to
read.
There are many excellent editors; one is Emacs.\footnote{%
  It may come with your operating system or 
  see \protect\url{http://www.gnu.org/software/emacs}.}

Here is a first example of a \python{} program.
Start your editor, open a file called \path{test.py}, and enter these lines.
(Notice that the first line describes the program in a comment.
You may want to also name the author, the date, and the license for the code.
Also, the triple-quoted documentation string at the top of the file
is a good practice.
You should include such a string in each program.)
\begin{lstlisting}[style=python]
# test.py
"""A first Python program. 
"""
import datetime
current = datetime.datetime.now()  # get a datetime object
print("The month number is", current.month)
\end{lstlisting}
Run it through \python{},
One way is to open a command line terminal,
change to the directory containing \path{test.py},  
type \inlinecode{python3 test.py}, and hit \keyboardkey{Enter}. 
You should see
output like \inlinecode{The month number is 9}.

Next is a small game. 
(It uses the \python{} function \inlinecode{input} that prompts the user
and then collects their response.).
\lstinputlisting{python/guessing_game.py}
Here is output from running this game.
\begin{lstlisting}
$ python3 guessing_game.py
Guess an integer between 1 and 10: 3
  Sorry, your guess is too low
Guess an integer between 1 and 10: 8
  Sorry, your guess is too high
Guess an integer between 1 and 10: 5
  You are right!
\end{lstlisting}  % $

As earlier, note the triple-quoted documentation strings both for the 
file as a whole and for the function.
Besides being useful to a programmer editing the file, 
they integrate with \python's help system. 
Go to the directory containing \path{guessing_game.py} and start 
the \python{} interpreter.
At the \inlinecode{>>>} prompt enter \inlinecode{import guessing_game}.
You will play through a round of the game (there is a way to avoid this
but it doesn't matter here).
You are now using \inlinecode{guessing_game} as a module.
Type
\inlinecode{help("guessing_game")}.
You will see documentation on \inlinecode{guessing_game}, 
including these lines
(to quit the viewer use \keyboardkey{Ctrl}-\textit{q}).
\begin{lstlisting}
Help on module guessing_game:

NAME
    guessing_game - guessing_game.py

DESCRIPTION
    Guess the number between 1 and 10.

FUNCTIONS
    test_guess(guess)
        Decide if the guess is correct and print a message.
\end{lstlisting}
\python{} got this information from the documentation strings.
\Sage's programmers follow this practice and so 
you can use \inlinecode{help(...)} 
whenever you need more information on one of 
\Sage's routines.





%----------------------------------
\section{\Sage}
Learning about \Sage's capabilities is the goal of much of this manual.
But \Sage{} is built on \python{} and so a brief comparison here makes sense.
See \citep{SageTeam19} for a more broad-based introduction, and 
be aware of the comprehensive reference \citep{SageTeam19ref}.



\subsection{Command line}
\Sage's command line is like \python's but adapted to 
mathematical work.
First start \Sage,
for instance, enter \inlinecode{sage} into a command line window.
You get some initial text and a prompt.
\begin{lstlisting}[style=python]
sage:  
\end{lstlisting}
Leave the prompt with the command \inlinecode{exit}
followed by \keyboardkey{Enter}.

You can use this interpreter to experiment with some expressions.
\begin{sagecommandline}
sage: 2**3                                                                      
sage: 2^3
sage: 3*1 + 4*2
sage: 5 == 3+3
sage: sin(pi/3)
\end{sagecommandline}
The second expression 
shows that \Sage{} provides a convenient shortcut for exponentiation over
\python's \inlinecode{2**3}.
The final one
shows that \Sage{} sometimes returns exact results rather than an
approximation.
You can still get the approximation; here are three ways.
\begin{sagecommandline}
sage: sin(pi/3).numerical_approx()
sage: sin(pi/3).n()
sage: n(sin(pi/3), digits=4)  
\end{sagecommandline}
The final one shows that there is an option to limit the number of digits
that you see.\footnote{%
  If you don't use the option then you see the number of digits matching 
  $53$ bits of precision, which is the standard for  
  floating point numbers.}


\subsection{Script}
You can group \Sage{} commands together in a file.
This way you can test the commands, 
and also reuse them without having to retype.

Create the \path{sage_normal.sage}, 
enter this function and save the file.
\lstinputlisting{sage/normal_curve.sage}
Start \Sage{} and bring the file in with \inlinecode{load(...)}.
\begin{sagecommandline}
sage: load("normal_curve.sage")
sage: normal_curve(1.0)
\end{sagecommandline}


\subsection{Notebook}
In this manual we will stick to the interpreter.
But we'll mention that another way to use \Sage{} is in   
a browser-based graphical interface, 
inside a Jupyter Notebook, \cite{JupyterTeam19}.
You can set up
worksheets to use alone or with other people, you can easily
go back and edit prior commands, view plots integrated with the text, 
and many other nice features.
To fire it up, instead of just issuing the \inlinecode{sage}
command, say
\inlinecode{sage -n jupyter}
(of course, you must have the Jupyter software
installed).\footnote{%
  If your operating system doesn't offer it then you can 
  visit \protect\url{https://jupyter.org}}
\endinput
